\chapter{转动及其他对称性算符}

\section{张量算符及Wigner-Eckart定理}

\subsection{Wigner-Eckart定理}
$T_{LM}$为$L$阶球张量,考虑这样一个量子态$\ket{\xi j m}$,角动量$j$及其第三分量$m$是好量子数,$\xi$包含了确定系统所需的其他所有量子数。这样,$T_{LM}$对应的矩阵元$\braket{\xi^{\prime} j^{\prime} m^{\prime} | T_{LM} | \xi j m}$可用Wigner-Eckart定理把角动量第三分量提取出来
\begin{equation}
    \boxed{
    \begin{aligned}
        \braket{\xi^{\prime} j^{\prime} m^{\prime} | T_{LM} | \xi j m}
        =& \hat{j^{\prime}}^{-1}\braket{jmLM | j^{\prime} m^{\prime}}
        \left(\xi^{\prime} j^{\prime} \| \bm{T}_{L} \| \xi j \right) \\
        =& (-1)^{j^\prime - m^\prime}
        \begin{pmatrix}
            j^{\prime} & L & j \\
            -m^{\prime}& M & m
        \end{pmatrix}
        \left(\xi^{\prime} j^{\prime} \| \bm{T}_{L} \| \xi j \right)      
    \end{aligned}
    }
    \label{eq:wigner-eckart}
\end{equation}
此处,$\left(\xi^{\prime} j^{\prime} \| \bm{T}_{L} \| \xi j \right)$为约化矩阵元,$\braket{\xi^{\prime} j^{\prime} m^{\prime} | T_{LM} | \xi j m}$为矩阵元。

\subsection{约化矩阵元的计算}
计算\cref{eq:wigner-eckart}定理中的约化矩阵元,只需取一些特定的值将左边的矩阵元算出,即可求出约化矩阵元。 


