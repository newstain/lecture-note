\chapter{波函数与波动方程}

\section{波函数}
经典物理中的粒子或系统是人们可以感知到的实物,但量子体系中的粒子则不然,例如电
子和原子核,我们从未实际观测到这些粒子,只能通过一些间接物理量来确定其真实存在。

关于“波函数”$\phi(\bmr, t)$的物理意义是什么这个问题,历史上曾有两种不同的看
法:薛定谔认为“波函数”就是经典意义上的波,与光波这样的性质是一致的;而以Niels 
Bohr为首的哥本哈根学派认为“波函数”表征“概率波”。后一种解释被普遍接受。而最充分
、最细致、最客观地论述波函数统计解释的德国科学家Max Born在提出概率波概念28年
后的1954年获得了诺贝尔物理学奖。

确切地说,波函数的平方$|\phi(\bmr, t)|^2$表示任意时刻$t$粒子在空间$\bmr$处的
单位体积中出现的概率,即概率密度(probability density)。\CJKunderdot{波函
数描述的是粒子在空间中的一种分布,而无法确切知道粒子在某一点的位置,换句话说,
我们无法确定微观粒子是沿着一个确定轨道运动的}。从这一点来看,粒子既具有实物粒子
的性质,表现为粒子具有质量、电荷等物理量上,同时粒子也具有波动的性质,具体形式表
现为“波函数”的概率性使粒子的位置无法确定,而是表现出与波一样的性质。也就是微观
粒子具有\CJKunderline{波粒二象性}。

\subsection{波函数的性质}
考虑到波函数描述的是一个微观粒子,而微观粒子在全空间内不会凭空消失也不会凭空产
生,因此在全空间内肯定能找到这个粒子,也就是说对这个粒子的概率密度进行全空间积
分等与$1$,这就是波函数的\CJKunderline{归一化条件}:
\begin{equation}
    \int_{-\infty}^{+\infty} | \phi | ^2 d\,x = 1
    \label{eq:wv-norm-condi}
\end{equation}

归一化条件\eqref{eq:wv-norm-condi}说明波函数$\phi$是平方可积的函数(square
-integrable function),这要求波函数满足
\begin{equation}
    \lim_{x \to \pm \infty} \phi \to 0
    \label{eq:wave-}
\end{equation}
否则对波函数平方的全空间求积分将导致无穷大,也就是说这个粒子到处分布在整个宇宙
中,这很明显是错的。

\section{Schr{\"o}dinger方程}
\section{Klein-Gordon方程}
\section{Dirac方程}