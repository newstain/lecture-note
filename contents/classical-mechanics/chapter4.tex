\chapter{刚体运动的动能}

\section{刚体的独立坐标}

刚体系统具有$N$个粒子,这些粒子间的相对位置是确定的,也就是说对于任意两个粒子$i$和$j$,其
相对距离满足
\begin{equation}
	r_{ij} = c_{ij}	\label{eq:rigid-constraint}
\end{equation} 
其中$c_{ij}$是一个常数。这是$N$粒子刚体系统满足的约束方程。确定这样一个系统的空间状态,我
们需要确定$N$个粒子的自由度,整个系统总共就有$3N$个自由度,这是非常复杂的,而上式所确定的
约束方程总共有$\dfrac{N(N-1)}{2}$个,对于很大的$N$来说,这远远大于了系统的所有粒子的自由度
数目,因此上述的约束方程并不是完全独立的。

在确定整个刚体系统的空间位置前,我们先考虑如何确定一个点在空间中的位置。对于空间中一个点$4$
的位置,我们可以先确定空间中不同直线上的三个点$1$、$2$、$3$的位置,并且确切地知道点$4$相对
于这三个点的距离时,我们就可以唯一确定点$4$的位置。现在考虑多粒子系统的刚体系统,我们可以
知道任意一点$i$相对于$1$、$2$、$3$的距离$c_{i1}$、$c_{i2}$、$c_{i3}$,这样,我们基于这三个
点就可以确定刚体内其余所有点的位置,也就确定了这个刚体的空间位置。

现在,我们已经把整个刚体的空间位置由$N$个点的自由度约化成了确定$3$个点的自由度问题,下面来
定出这$3$个点所需要的自由度个数。首先,我们知道了刚体内所有点的约束条件为式\eqref{eq:rigid-constraint},
从而知道这三个点具有确定的相对距离为
\begin{equation*}
	r_{12} = c_{12}, \quad r_{13} = c_{13}, \quad r_{23} = c_{23}
\end{equation*} 
我们确定点$1$需要三个自由度;点$2$在以点$1$为球心,$r_{12}$为半径的球面上,因此我们只需要
知道点$2$的两个自由度便可以确定其位置;在确定前两个点的位置后,点$3$位于点$1$和点$2$为球心,
$r_{13}$和$r_{23}$为半径的球面相交所构成的圆上,因此,我们只需要确定其中一个自由度就可以知
道点$3$的确定位置。综上,我们总共需要确定$3+2+1=6$个自由度便可以最终定下三个点的具体空间位
置,再根据刚体内点的约束关系,就可以确定好整个系统的空间位置。

\paragraph*{相对坐标系统}
\begin{figure}[htbp]
	\centering
	\begin{tikzpicture}
    \draw[-Stealth, line width=1pt]             (0, 0) -- (4, 0);
    \draw[-Stealth, line width=1pt, rotate=90]  (0, 0) -- (4, 0);
    \draw[-Stealth, line width=1pt, rotate=225]  (0, 0) -- (3, 0);

    \node at (4, 0)  [below] {$y$};
    \node at (0, 4)  [left] {$z$};
    \node at (-2, -2) [left] {$x$};
\end{tikzpicture}

	\label{fig:relative-coord}
\end{figure}
空间中有一套Cartesian坐标系统,刚体处于这个坐标系统中,称为unprimed坐标系统,单位矢量为
$\vbe_{i}$、$\vbe_{j}$和$\vbe_{k}$;在刚体内部建立另一套Cartesian坐标系统,刚体内部的粒子用此坐标系统来标记,称为primed坐标系统,其单位矢量为$\vbe_{i}^{\,\prime}$、$\vbe_{j}^{\,\prime}$和$\vbe_{k}^{\,\prime}$。那么,unprimed和primed就是两套相对的坐标系统,若我们需要知道刚体中某个粒子的状态,我们只需要知道这个粒子在primed坐标系中的状态,再知道primed相对于unprimed坐标系中的状态,就可以确定刚体中的粒子运动。

矢量不会因为我们改变坐标系统而改变其空间分布,因此两个坐标系统中观察到的是同一个矢量,故
\begin{equation}
	\vbr = x \vbe_{i} + y \vbe_{j} + z \vbe_{k}
	     = x^{\prime} \vbe_{i}^{\,\prime} + y^{\prime} \vbe_{j}^{\,\prime} + z^{\prime} \vbe_{k}^{\,\prime}
\end{equation}
利用上式,对矢量的描述中,从unprimed坐标变换到primed坐标的矢量分量变换为:
\begin{equation}
	\begin{aligned}
		x^{\,\prime} =& \cos{\theta_{11}} x + \cos{\theta_{12}} y + \cos{\theta_{13}} z \\
		y^{\,\prime} =& \cos{\theta_{21}} x + \cos{\theta_{22}} y + \cos{\theta_{23}} z \\
		z^{\,\prime} =& \cos{\theta_{31}} x + \cos{\theta_{32}} y + \cos{\theta_{33}} z
	\end{aligned}
\end{equation}
由正交关系($\alpha$、$\beta$遍历$i$、$j$、$k$)
\begin{equation}
	\vbe_{\alpha} \cdot \vbe_{\beta} = \delta_{\alpha\beta}
	\quad
	\vbe_{\alpha}^{\,\prime} \cdot \vbe_{\beta}^{\,\prime} = \delta_{\alpha\beta}^{\,\prime}
\end{equation}
并结合以上两式,我们可以得到两坐标系统单位矢量间的变换关系:
\begin{enumerate}
	\item 从primed坐标变换到unpried坐标,满足
		\begin{equation}
			\begin{aligned}
				\vbe_{i}^{\,\prime} =& \cos{\theta_{11}} \vbe_{i} + \cos{\theta_{12}} \vbe_{j} + \cos{\theta_{13}} \vbe_{k} \\
				\vbe_{j}^{\,\prime} =& \cos{\theta_{21}} \vbe_{i} + \cos{\theta_{22}} \vbe_{j} + \cos{\theta_{23}} \vbe_{k} \\
				\vbe_{k}^{\,\prime} =& \cos{\theta_{31}} \vbe_{i} + \cos{\theta_{32}} \vbe_{j} + \cos{\theta_{33}} \vbe_{k}
			\end{aligned}
		\end{equation}
	\item 从umprimed坐标变换到primed坐标,满足
	\begin{equation}
		\begin{aligned}
			\vbe_{i} =& \cos{\theta_{11}} \vbe_{i}^{\,\prime} + \cos{\theta_{21}} \vbe_{j}^{\,\prime} + \cos{\theta_{31}} \vbe_{k}^{\,\prime} \\
			\vbe_{j} =& \cos{\theta_{12}} \vbe_{i}^{\,\prime} + \cos{\theta_{22}} \vbe_{j}^{\,\prime} + \cos{\theta_{32}} \vbe_{k}^{\,\prime} \\
			\vbe_{k} =& \cos{\theta_{13}} \vbe_{i}^{\,\prime} + \cos{\theta_{23}} \vbe_{j}^{\,\prime} + \cos{\theta_{33}} \vbe_{k}^{\,\prime}
		\end{aligned}
	\end{equation}
	\item 方向角的正交关系为
		\begin{equation}
			\sum_{l = 1}^{3} \cos{\theta}_{lm^{\prime}} \cos{\theta}_{lm} = \delta_{mm^{\prime}}
		\end{equation}
\end{enumerate}

\paragraph*{正交变换}
现在,用$x_1$、$x_2$、$x_3$分别表示$x$、$y$和$z$,方向余弦重新表述如下
\begin{equation}
	a_{ij} = \cos{\theta_{ij}}
\end{equation}
%%%%%%%%%%%%%%%%%%%%%%%%
\section{欧拉角}
欧拉角用于描述一个物体定点转动。转动矩阵的行列式的值为$+1$。

%%%%%%%%%%%%%%%%%%%%%%%%
\section{无穷小转动}
