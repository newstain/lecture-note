\chapter{力学基本原理}

%%%%%%%%%%%%%%%%%%%%%%%%%%%%%%%%%%%%%%%%%%%%%%%%%%%%%%%%%%%%%%%%%%%%%%%%%%%%%%%%%%
\section{单粒子力学原理}
\begin{theorem}[单粒子守恒原理]
	线动量守恒:若合力$\bm{F} = 0$, 即$\bm{\dot{p}} = 0$,则线动量$\bm{p}$守恒。
\end{theorem}
考虑一个单粒子运动,其位置矢量表示为$\bm{r}$,对应的速度为
\begin{equation}
  \bm{v} = \frac{d \bm{r}}{d t} \label{eq_single_velocity}
\end{equation}
粒子质量为$m$,则其动量定义为:
\begin{equation}
  \bm{p} = m \bm{v}   \label{eq_single_momentum}
\end{equation}
其方向与速度方向一致。由于粒子可能与外部物体或场发生相互作用,因此粒子会受到外部力的作用,粒子所受到的合力可以写为;
\begin{equation}
  \bm{F} = \frac{d \bm{p}}{dt} = \dot{\bm{p}} = \frac{d }{dt}(m\bm{v})  \label{eq_single_tot_force}
\end{equation}
在质量守恒的情况下,上式可写为:
\begin{equation}
  \bm{F} = m \frac{d\bm{v}}{dt} m \bm{a}
\end{equation}
其中$\bm{a}$为粒子的矢量加速度,其定义为
\begin{equation}
	\bm{a} = \frac{d^2 \bm{r}}{dt^2}
\end{equation}
此运动方程仅为二阶微分方程,假设$\bm{F}$并不依赖于高阶微分项。
\begin{definition}[惯性系(伽利略系)]
	在惯性系中,对于不受合外力影响的物体将保持相对静止或匀速直线运动,其时间是均匀流逝的,空间也是均匀且各向同性的。此参考系使方程\eqref{eq_single_tot_force}成立。
\end{definition}

\vspace{5pt}
\begin{theorem}[角动量守恒原理]
	若一物体的总扭矩$\bm{N}=0$,即角动量$\bm{L}$对时间的一次微分项$\dot{\bm{L}} = 0$,则粒子的角动量$\bm{L}$守恒,也就是总角动量$\bm{L}$不随时间变化。
\end{theorem}
若粒子绕某一点$O$旋转,其与点$O$的径矢为$\bm{r}$,动量为$\bm{p}$,可定义其角动量为
\begin{equation}
	\bm{L} = \bm{r} \times \bm{p}	\label{eq_angular_momentum}
\end{equation}
类似公式\eqref{eq_single_tot_force},定义该粒子关于某一点的\underline{\textit{力矩}}为
\begin{equation}
	\bm{N} = \frac{d\bm{L}}{dt} = \frac{d \bm{r}}{dt} \times \bm{p} + \bm{r} \times \frac{d\bm{p}}{dt} = \bm{v} \times (m\bm{v}) + \bm{r} \times \frac{d\bm{p}}{dt} = \bm{r} \times \frac{d(m\bm{v})}{dt} = \bm{r} \times \bm{F}
\end{equation}
显然,我们可以得到力矩与角动量之间的关系为:
\begin{equation}
	\bm{N} = \frac{d\bm{L}}{dt} = \dot{\bm{L}}
\end{equation}
\begin{note}
	需要注意的是,$\bm{L}$和$\bm{N}$都依赖于点$O$。
\end{note}

\vspace{5pt}
\begin{theorem}[能量守恒原理]
	若作用在一个粒子上的力是保守的,那么这个粒子的总能量(总动能+势能) $T+V$是守恒的。
\end{theorem}
考虑一个粒子在外力$\bm{F}$作用下从点$1$到点$2$所做的功。
\begin{equation}
	W_{12} = \int_{1}^{2} \bm{F} \cdot d\bm{s}	\label{eq_work_done}
\end{equation}
若质量为常量(若无特殊说明,第一部分中质量皆为常数),则方程\eqref{eq_work_done}可约化为:
\begin{equation}
	\int_{1}^{2} \bm{F} \cdot d\bm{s} = m \int \frac{d\bm{v}}{dt}\bm{v} dt = \frac{m}{2} \int \frac{d}{dt}(v^2) dt = \frac{m}{2} \int \frac{d(v^2)}{dt} 
\end{equation}
因此,我们有
\begin{equation}
	W_{12} = \frac{m}{2}(v_2^2 - v_1^2)
\end{equation}
其中,标量$mv^2/2$表示该粒子的动能,并用$T$标记,则上式可改写为
\begin{equation}
	W_{12} = T_2 - T_1	\label{eq_tot_different_kinetic}
\end{equation} 
上式表示在外力$\bm{F}$的作用下,粒子从位置1到位置2的动能变化。
\begin{note}
	\underline{力场保守},也就是该力场下使粒子从点1沿任意路径运动到点2,其所作功$W_{12}$不变;换种说法,若力场保守,则粒子从起点出发沿任意闭合路径在运动回起点,所作功为0,即满足下式
	\begin{equation}
		\oint \bm{f} d\bm{s} = 0	\label{eq_force_circuit_path}
	\end{equation} 
	可类比于重力场,我们在不考虑其他外力的作用下,一个物体从地面运动到楼顶在回到地面,重力对它是不做功的。
\end{note}
$W_{12}$与路径无关的的充要条件是该粒子受力为某一标量坐标函数的梯度:
\begin{equation}
	\bm{F} = - \nabla V(\bm{r})	\label{eq_force_derived_potential}
\end{equation} 
其中,$V$就是所谓的势,或者称为势能。若$W_{12}$与路径无关,仅与两个终点有关,那么它对路径的微分应该表现为一个常量,这个常量应该可以由$-V$对路径的微分进行确定,或该常量与路径微分的标积等于$-V$的微分:
\begin{equation*}
	\bm{F} \cdot d \bm{s} = -d V  \quad \text{or} \quad F_{s} = \frac{\partial V}{\partial s}
\end{equation*} 
上式与方程\eqref{eq_force_derived_potential}是一致的。
\begin{note}
	注意到\Cref{eq_force_derived_potential},方程左边加入任意一个常量其结果不变,因此可以知道,零势能面是可以任意确定的。
\end{note}
对于一个守恒的系统,从点1到点2的所作的功可用势能之差表示,如下:
\begin{equation}
	W_{12} = V_2 - V_1	\label{eq_work_down_v}
\end{equation} 
由式\eqref{eq_tot_different_kinetic}和\eqref{eq_work_down_v}可得如下关系:
\begin{equation}
	T_1 + V_1 = T_2 + V_2	\label{eq_energy_conservation}
\end{equation} 
上式表示,在一个守恒体系下,系统在某一点出的动能与势能之和等于某一常数。

%%%%%%%%%%%%%%%%%%%%%%%%%%%%%%%%%%%%%%%%%%%%%%%%%%%%%%%%%%%%%%%%%%%%%%%%%%%%%%%%%%
\section{约束}
\paragraph*{广义坐标}
我们需要先了解一下什么是\underline{自由度}。在笛卡尔坐标系中,描述一个自由粒子的确切位置可用$(x,y,z)$三个坐标来标定,这三个坐标之间无关联关系,我们成这个粒子具有三个自由度;此时,我们再考虑一个具有$N$个自由粒子的系统,此时每个粒子都有一套独立的坐标$(x_1, y_1, z_1), (x_2, y_2, z_2), \cdots (x_n, y_n, z_n)$一共$3N$个独立坐标来标定整个系统的状态,则我们称这个系统具有$3N$个自由度。现在,我们对这个系统施加$k$个完整约束,则对整个系统有$k$个约束方程,对这些方程进行联立则可消掉$k$个自由度,则整个系统只剩下$3N-k$个独立变量,我们将其称为此系统的$3N-k$个自由度,我们用$q_1, q_2, \cdots, q_{3N-k}$来表示,则每个粒子的坐标可以表示为:
\begin{equation*}
	\begin{aligned}
		\bm{r}_1 &= \bm{r}_1(q_1, q_2, \cdots, q_{3N-k})	\\
					   &\cdots	\\
		\bm{r}_N &= \bm{r}_N(q_1, q_2, \cdots, q_{3N-k})	\\
	\end{aligned}
\end{equation*} 
上式称为从$\bm{r}_i$到$q_i$的转换方程。
\begin{note}
	广义坐标一般与三维笛卡尔坐标是不一样的,他们不一定正交,只是互不相关的一些量而已。	
\end{note}

\paragraph*{完整约束(holonomic)}
完整约束具有以下形式
\begin{equation}
    f(\bm{q}, t) = 0	\label{eq:constrain_holo}
\end{equation} 
例如刚体,刚体系统中每个点之间的距离是固定的,有如下形式:
\begin{equation*}
    (\bm{r}_i - \bm{r}_j)^2 - c_{ij}^2 = 0
\end{equation*} 
此形式的方程就是具有完整约束的约束方程。对于其他一些约束,形如:
\begin{equation}
    f(\bm{q}, \dot{\bm{q}}, t) = 0
    \label{eq:eq:constrain-holo-disg}
\end{equation}
若通过积分将上式化成如式\eqref{eq:constrain_holo}的形式,则称这样的约束为隐藏的完整约束
(holonomic constraint in disguise)。

\paragraph*{非完整约束(nonholonomic)}
无法将约束形式写为或者划归成方程\eqref{eq:constrain_holo}的约束称为非完整约束,例如在空心
球内的气体运动:
\begin{equation*}
	\bm{r} - a^2 = 0
\end{equation*} 
此外,还可根据约束中是否包含时间来进行划分,约束中显示的包含时间则将此约束类型称为非恒稳约束(rheonomous),约束中不包含时间则为恒稳约束(scleronomous)。

此外,约束也可能带来一些问题,如下:
\begin{enumerate}
	\item 受约束方程的影响,粒子坐标不再独立,而这会使每个粒子的运动方程发生联系;
	\item 在很多问题的求解中需要知道约束力,而这往往是未知的。
\end{enumerate}

%%%%%%%%%%%%%%%%%%%%%%%%%%%%%%%%%%%%%%%%%%%%%%%%%%%%%%%%%%%%%%%%%%%%%%%%%%%%%%%%%%
\section{作用力与约束力}
\paragraph*{作用力(Applied force)}
作用在系统的外部力称为作用力。

\paragraph*{约束(Constraint force)}
系统内部的力称为约束力。

%%%%%%%%%%%%%%%%%%%%%%%%%%%%%%%%%%%%%%%%%%%%%%%%%%%%%%%%%%%%%%%%%%%%%%%%%%%%%%%%%%
\section{虚位移,虚功原理和达朗贝尔原理}

%%%%%%%%%%%
\paragraph*{虚位移(virtual displacement)}
虚位移也称无穷小位移(infinitesimal displacement),表示在某一时刻$t$,物体在约束条件下发生
的任意无穷小位移。虚位移与实际发生的位移是不一样的,它是物体在受约束情况下可能发生的任何
一种无穷小位移,但实际并没发生,而实际位移则是物体真实产生的位移,一旦发生便无法改变。举个例子,若一个物体被约束在$xOy$平面上运动,则其虚位移如下:
\begin{figure}[htbp]
	\centering
	\setlength{\abovecaptionskip}{0.2cm}
	\begin{tikzpicture}[auto]
		\draw	(0, 0)	circle	[radius=1cm];
		\filldraw	[gray]	(0, 0)	circle	[radius=0.1cm];
		\draw	[-Latex]	(0, 0)	-- (0, 1);
		\draw	[-Latex]	(0, 0)	-- (1, 0);
		\draw	[-Latex]	(0, 0)	-- (-1, 0);
		\draw	[-Latex]	(0, 0)	-- (0, -1);
		\draw	[-Latex]	(0, 0)	-- (0.7071, 0.7071);
		\draw	[-Latex]	(0, 0)	-- (-0.7071, 0.7071);
		\draw	[-Latex]	(0, 0)	-- (0.7071, -0.7071);
		\draw	[-Latex]	(0, 0)	-- (-0.7071, -0.7071);
	\end{tikzpicture}
	\caption{虚位移$\delta \bm{r}$的图像,此时约束力垂直平面向里。}
\end{figure}
物体能在$xOy$平面内任意移动,但无法沿$z$轴方向移动。

%%%%%%%%%%%
\paragraph*{虚功原理}
为推导此原理,我们假设系统处于平衡状态,也就是作用在系统中每一粒子的合力$\bm{F}_i = 0$,
因此,系统中每一点的虚功(合外力在虚位移上所做的功)为0,即$\bm{F}_i \cdot \delta \bm{r}_i =0$。对系统中所有粒子的虚功取和,我们同样应该由
\begin{equation}
	\sum_i \bm{F}_i \cdot \delta \bm{r}_i  = 0 \label{eq_system_virtual_work}
\end{equation} 
系统中某一粒子的合力由作用力$\bm{F}^{(a)}_i$和约束力$\bm{f}_i$,对此,式\eqref{eq_system_virtual_work}可化为:
\begin{equation}
	\sum_i \bm{F}^{(a)}_i \cdot + \sum_i \bm{f}_i \cdot \delta i = 0
                                  \label{eq_system_virtual_work_decomposed}
\end{equation} 
现在,我们仅考虑净虚功为0的情况,也就是约束力对所有虚位移所作功之和为0的情况:比如一个物
体被约束在某一平面,那么约束力与该平面垂直,而虚位移只能与该平面相切,此时的虚功便为0。在
这种情况下,式\eqref{eq_system_virtual_work_decomposed}左边第二项变为0,进一步,我们可以知
道其作用力在虚位移上所作的功也为0:
\begin{equation}
	\sum_i \bm{F}^a_i \cdot \delta \bm{r}_i = 0	\label{eq_virtual_work_principle}
\end{equation} 
式\eqref{eq_virtual_work_principle}就是所谓的\underline{\textcolor{red}{虚功原理}}。由于$\bm{r}_i$受约束作用,它们并不独立,因此我们一般有$\bm{F}^{(a)}_i \neq 0$,因此,在具体计算时,我们需要将上述虚位移转化为相互独立的广义坐标$q_i$空间进行计算。
\begin{note}
	若一个粒子被约束在某一平面且存在摩擦,由于约束力总与虚位移垂直,某一时刻的物体在虚位移
    上所作的虚功仍为0;但是,物体在真实位移上所作功(并不特指约束力所作功)不一定为0,因为摩
    擦力会作功。
\end{note}

%%%%%%%%%%%
\paragraph*{达朗贝尔原理(D'Alembert's principle)}
上述虚功原理只适合用于系统平衡的情况,当系统处于运动中时,需要使用达朗贝尔原理进行计算。
一般的,\textcolor{red}{ 运动方程 }为:
\begin{equation*}
	\bm{F}_i = \dot{\bm{p}}_i
\end{equation*} 
可将上式重新写为:
\begin{equation*}
	\bm{F}_i - \dot{\bm{p}} = 0
\end{equation*}
上式$\bm{F}_i$表示作用在一个粒子上的合外力,减去$\dot{\bm{p}}$后整个式子变为了0,我们将$-\dot{\bm{p}}$称为“相反的有效力”,它与合外力共同作用使此粒子处于平衡态。将上式代入\Cref{eq_system_virtual_work},并遍历所有粒子,我们有:
\begin{equation}
	\sum_{i}(\bm{F}_i - \dot{\bm{p}}_i) \cdot \delta \bm{r}_i = 0
\end{equation} 
同时,我们按式\eqref{eq_virtual_work_principle}将合力分解为作用力和粒子间作用力,得到
\begin{equation*}
	\sum_i (\bm{F}^{(a)}_i - \dot{p}_i) \cdot \delta \bm{r}_i + \sum_i \bm{f}_i \cdot \delta \bm{r}_i = 0)
\end{equation*} 
同样,我们只在讨论约束力所作虚功为0的情况,则上式最后可以化为:
\begin{equation}
	\sum_i (\bm{F}^{(a)}_{i} - \dot{\bm{p}}_i) \cdot \delta \bm{r}_i = 0	\label{eq:dalember_principle}
\end{equation} 
称\Cref{eq:dalember_principle}为\underline{达朗贝尔原理(D'Alembert's principle)}。现在,我们知道,当约束力所作虚功为0时,合外力所作功与物体运动加速度产生的所谓“相反有效力”所作功相等,此处,我们可以将上标$^{(a)}$去掉。

%%%%%%%%%%%
\paragraph*{广义坐标下的达朗贝尔原理}
为方便求解动力学问题,我们需要将达朗贝尔原理用广义坐标来表示。我们将$\bm{r}_i$转换为广义坐
标$q_{j}$(假设有$n$个独立坐标),其变换方程为:
\begin{equation*}
	\bm{r}_i = \bm{r}_i(q_1, q_2, \cdots , q_n, t)
\end{equation*} 
这样,$\bm{r}_i$对应的速度可以表示为:
\begin{equation}
	\bm{v}_i \equiv \frac{d\bm{r}_i}{dt} = \sum_k \frac{\partial \bm{r}_i}{\partial q_k}\dot{q}_k + \frac{\partial \bm{r}_i}{\partial t}
	\label{eq:velocity-gener-coord}
\end{equation} 
类似地,我们可以得到虚位移$\delta\bm{r}_i$的表示:
\begin{equation}
	\delta\bm{r}_i = \sum_j \frac{\partial \bm{r}_i}{\partial q_j}\delta q_j
	\label{eq:virtual-displace-expand}
\end{equation} 
\begin{note}
	虚位移是在某一时刻下物体可能发生的任意无穷小的位移,因此它只是广义坐标的函数,而不包
    含时间的变化$\delta t$。此外,我们在讨论虚位移时确实也只应该将其当成广义坐标的函数,
    应该排除其包含时间,因此当约束力随时间变化时,虚位移与约束力应当是相互垂直的。
\end{note}

%%%%%%%%%%%
\paragraph*{广义力}
在广义坐标下,$\bm{F}_i$所作的功表示为
\begin{equation}
	\sum_i \bm{F}_i \cdot \delta \bm{r}_i = \sum_{i, j}\bm{F}_i \cdot \frac{\partial \bm{r}_i}{\partial q_j} \delta q_j = \sum_{j} Q_j \delta q_j
\end{equation} 
其中,$Q_j$就是所谓的\textcolor{red}{广义力的组分},其形式为
\begin{equation}
	Q_j = \sum_{i} \bm{F}_i \cdot \frac{\partial \bm{r}_i}{\partial q_j}	\label{eq_generalized_force}
\end{equation} 
\begin{note}
	广义坐标$q$单位不一定为长度,同样地,广义力$Q$的单位也一定为力的单位,但必须保证他们
    的乘积具有功的单位。如$q$表示转动的角度$d\theta$而$Q$表示力矩$N_j$,他们的乘积
    $N_j d\theta$则具有功的单位。
\end{note}

%%%%%%%%%%%
\paragraph*{达朗贝尔原理用广义力和广义动量表示}
根据\Cref{eq:dalember_principle,eq:virtual-displace-expand},将其中的有效力$\dot{\bm{p}}$改写为
\begin{equation*}
	\begin{aligned}
		\sum_{i} \dot{\bm{p}}\cdot \delta\bm{r}_i = \sum_{i} m_i \ddot{\bm{r}}_i \cdot\delta\bm{r}_i = \sum_{ij} m_i \ddot{\bm{r}}_i \cdot\frac{\partial \bm{r}_i}{\partial q_j}\delta q_j
	\end{aligned}
\end{equation*}
现在利用关系式
\begin{equation}
	\sum_{i} m_i \ddot{\bm{r}}_i\cdot\frac{\partial\bm{r}_i}{\partial q_j}
	=\sum_{i}\left[\frac{d}{dt}\left(m_i\dot{\bm{r}}_i\cdot\frac{\partial\bm{r}_i}{\partial q_j}\right) - m_i\dot{\bm{r}}_i\cdot\frac{d}{dt}\left(\frac{\partial\bm{r}_i}{\partial q_j}\right)\right]
	\label{eq:math-exp}
\end{equation}
上式左端最后一项交换对时间$t$和广义坐标$q_j$的偏分,这样有
\begin{equation}
	\begin{aligned}
	\frac{d}{dt}\left(\frac{\partial\bm{r}_i}{\partial q_j}\right)
	=\frac{\partial\dot{\bm{r}}_i}{\partial q_j}
	=&\frac{\partial}{\partial q_j} \frac{d\bm{r}}{dt} \\
	=&\frac{\partial}{\partial q_j} 
	      \left[\sum_{k}\left(\frac{\partial\bm{r}_i}{\partial q_k}\frac{d q_{k}}{dt} + \frac{\partial \bm{r}_i}{\partial t}\frac{\partial t}{\partial t}\right)\right]
	=\sum_{k}\frac{\partial^2\bm{r}_i}{\partial q_j\partial q_k}\dot{q_k} + \frac{\partial^2 \bm{r}_i}{\partial q_j \partial t}	 \\
	=&\frac{\partial \bm{v}_i}{\partial q_j}
\end{aligned}
\end{equation}
最后一个等式用了$\dot{\bm{r}}_i = \bm{v}_i$,第三个等式右边的第二项是因为坐标$\bm{r}$是广义坐标与时间的函数。由\Cref{eq:velocity-gener-coord}得
\begin{equation}
	\frac{\partial \bm{v}_i}{\partial \dot{q}_j} 
	= \sum_{k}\left(\dot{q}_j\frac{\partial}{\partial q_k}\cancel{\frac{\partial \bm{r}_i}{\partial \dot{q}_j}} + \frac{\partial \bm{r}_i}{\partial q_k}\frac{\partial \dot{q}_k}{\partial \dot{q}_j} + \frac{\partial}{\partial t}\cancel{\frac{\partial \bm{r}_i}{\partial \dot{q}_j}}\right)
	= \frac{\partial \bm{r}_i}{\partial q_j}
\end{equation}
上式第一个等号左边第一项和最后一项为0是因为坐标$\bm{r}_i$并非广义速度$\dot{q}_j$的函数,第二项当且仅当$j = k$时有$\partial \dot{q}_k / \partial \dot{q}_j = 1$,否则为0。将上式代入\Cref{eq:math-exp}中,有
\begin{equation}
	\begin{aligned}
	\sum_{i} m_i \ddot{\bm{r}}_i\cdot\frac{\partial\bm{r}_i}{\partial q_j}
	=& \sum_{i}\left[\frac{d}{dt}\left(m_i\dot{\bm{r}}_i\cdot\frac{\partial\bm{r}_i}{\partial q_j}\right) - m_i\dot{\bm{r}}_i\cdot\frac{d}{dt}\left(\frac{\partial\bm{r}_i}{\partial q_j}\right)\right] \\
	=& \sum_{i}\left[\frac{d}{dt}\left(m_i\bm{v}_i\cdot\frac{\partial\bm{v}_i}{\partial \dot{q}_j}\right) - m_i\bm{v}_i\left(\frac{\partial\bm{v}_i}{\partial \dot{q}_j}\right)\right] \\
	=& \sum_{i}\left[\frac{d}{dt}\left(\frac{\partial}{\partial \dot{q}_j} \frac{1}{2}m_i\bm{v}_i^2\right) -\frac{\partial\bm{v}_i}{\partial q_j}\frac{1}{2} m_i\bm{v}_i^2\right]
	\end{aligned}
\end{equation}
进一步,\Cref{eq:dalember_principle}中,作用力用广义力\Cref{eq_generalized_force}表示,则等号左边可以表示为
\begin{equation*}
	\sum_{j} \left\{\frac{d}{dt}\left[\frac{\partial}{\partial \dot{q}_j}\left(\sum_{i}\frac{1}{2}m_i v_i^2\right)\right] - \frac{\partial}{\partial q_j}\left(\sum_i \frac{1}{2}m_i v_i^2\right) - Q_j\right\}\delta q_j
\end{equation*}
上式中,$\sum_{i} \frac{1}{2}m_i v_i^2$为系统的动能,则达朗贝尔原理最终写为
\begin{equation}
	\sum \left\{ \left[\frac{d}{dt}\left(\frac{\partial T}{\partial \dot{q}_j}\right) - \frac{\partial T}{\partial q_j}\right] - Q_j \right\} \delta q_j = 0
	\label{eq:genera-dalember}
\end{equation}

\paragraph*{拉格朗日方程}
注意到\Cref{eq:genera-dalember}中的广义坐标$q_j$可以是任何能完整描述系统的一套坐标,而且它们应该互相独立,这样,\Cref{eq:genera-dalember}中$\delta q_j$前面的系数应等于0才能使等式成立,于是有
\begin{equation}
	\frac{d}{dt} \left(\frac{\partial T}{\partial \dot{q}_j}\right) - \frac{\partial T}{\partial q_j} = Q_j
	\label{eq:genera-kin-force}
\end{equation}
如果有$n$个独立坐标,那么就有$n$个这样的方程。此外,标量势能$V$对坐标的偏分为广义力,广义力可进一步用势能来表示
\begin{equation}
	Q_j = \sum_{i} \bm{F}_i \cdot \frac{\partial r_i}{\partial q_j} = \sum_{i} -\nabla_i V \cdot \frac{\partial \bm{r}_i}{\partial q_j} = - \sum_{i} \frac{V}{\partial \bm{r}_i} \frac{\partial \bm{r}_i}{\partial q_j} = - \frac{\partial V}{\partial q_j}
\end{equation}
上式第三个等式是由于势能是$\bm{r}_i$的函数$V = V(\bm{r}_1, \bm{r}_2, \cdots, \bm{r}_n, t)$。

上式代入\Cref{eq:genera-kin-force},有
\begin{equation}
	\frac{d}{dt} \left(\frac{\partial T}{\partial \dot{q}_j}\right) - \frac{\partial (T - V)}{\partial q_j} = 0
	\label{eq:pre-lagrangian-eq}
\end{equation}
当势能不显含时间时,系统守恒,且势能不依赖于广义速度,则定义拉格朗日量$L$为
\begin{equation}
	L = T - V
	\label{eq:lagrangian}
\end{equation}
这样,将其代回\Cref{eq:pre-lagrangian-eq}就能获得\CJKunderdot{拉格朗日方程}:
\begin{equation}
	\frac{d}{dt} \left(\frac{\partial L}{\partial \dot{q}_j}\right) - \frac{\partial (L)}{\partial q_j} = 0
\end{equation}