\chapter{数值积分}

\section{一维数值积分}
\subsection{辛普森积分及复合辛普森积分}
\paragraph*{辛普森积分} 区间$[a, b]$有函数$f(x)$,求函数在此区间内的积分,可定义此区间的中点$\dfrac{a+b}{2}$及对应的函数值$f(\dfrac{a+b}{2})$,也就是对区间$[a, b]$进行二等分,这样,对应的数值积分为
\begin{equation}
	I = \int_{a}^{b} f(x) \, dx \approx \frac{h}{6}\left[ f(a) + f(\frac{a+b}{2}) + f(b) \right]	\label{eq:simpson-integer}
\end{equation} 

\paragraph*{复合辛普森积分} 区间$[a, b]$内有函数$f(x)$,现将区间$n$等分,在区间$[x_{k}, x_{k+1}](k = 0, 1, 2, \cdots, n-1)$内应用式\eqref{eq:simpson-integer},并对每个小区间进行累加,这样就可以得到整个区间内的积分。设$x_{k+1/2} = \dfrac{x_k + x_{k+1}}{2}$,这样,$[a, b]$区间内对$f(x)$的积分为
\begin{equation}
    \begin{aligned}
		I =& \int_{a}^{b} f(x) \, dx = \sum_{i = 0}^{n - 1}\int_{x_i}^{x_{i+1}} f(x) \, dx	\\
		=& \frac{h}{6}\sum_{i=1}^{n-1} \left[ f(x_i) + 4f(x_{k+1/2}) + f(x_{k+1})  \right] + R_n(f)
    \end{aligned}
\end{equation} 
取近似
\begin{equation}
    \begin{aligned}
    	I \approx S_n = \frac{h}{6}\sum_{i=1}^{n-1} \left[ f(x_i) + 4f(x_{k+1/2}) + f(x_{k+1})  \right]
    \end{aligned}
\end{equation} 
上式称为\uline{复合辛普森积分}。余项为
\begin{equation}
	R_n(f) = I - S_n = -\frac{h}{180} \left(\frac{h}{2} \right) \sum_{i = 0}^{n-1}f^{(4)}(\eta_i), \quad \eta_i \in (x_i, x_{i+1})
\end{equation} 
其中,$h = x_{i+1} - x_{k}$,误差阶为$h^4$。

\begin{example}
	\textbf{c++实现复合辛普森积分:} 考虑一些数据点,
	
\end{example}

\subsection{Cubic积分}
在完成cubic插值后,我们可根据方程\eqref{eq:unkown_Sx}求出原函数的近似,区间$[x_i, x_{i+1}]$区间内有函数$f(x)$,此时,我们有
\begin{equation}
	\begin{aligned}
		I_i =& \int_{x_i}^{x_{i+1}} f(x) \, dx \approx \int_{x_i}^{x_{i+1}} S(x) \, dx \\
		=& \int_{x_i}^{x_{i+1}} \, dx \left[ M_i \frac{\left(x_{i+1}-x\right)^3}{6 h_i}+M_{i+1} \frac{\left(x-x_i\right)^3}{6 h_i}+\left(y_i-\frac{M_i h_i^2}{6}\right) \frac{x_{i+1}-x}{h_i} \right. \\
		 &\left. + \left(y_{i+1}-\frac{M_{i+1} h_i^2}{6}\right) \frac{x-x_i}{h_i} \right]	\\
		=& \left[ -\frac{M_i}{24h_i} \left(x_{i+1}-x\right)^4 + \frac{M_{i+1}}{24h_i}\left(x-x_i\right)^4 \right. \\
		 &\left. -\left(y_i-\frac{M_i h_i^2}{6}\right) \frac{(x_{i+1}-x)^2}{2h_i} + \left(y_{i+1}-\frac{M_{i+1} h_i^2}{6}\right) \frac{(x-x_i)^2}{2h_i} + \text{Const} \right]_{x_i}^{x_{i+1}}	\\
		=& \frac{y_i + y_{i+1}}{2} h_i - \frac{M_{i} + M_{i+1}}{24} h_i^3
	\end{aligned}
\end{equation} 
其中,$h_i = x_{i+1}-x_{i}$。区间$[a,\ b]$存在函数$f(x)$,将此区间进行$n$等分,每个小区间
为$a = x_0 < x_1 < x_2 < \cdots < x_{n-1} < x_n = b $,求函数在此区域内的积分,如下
\begin{equation}
	I = \sum_{i=0}^{n-1} I_{i} \approx \sum_{i=0}^{n-1} \frac{y_i + y_{i+1}}{2} h_i - \frac{M_{i} + M_{i+1}}{24} h_i^3
\end{equation} 
当为等距格点时,即$h = h_0 = h_1 =\cdots = h_{n-1}$时,上式化为
\begin{equation}
	I\approx\frac{y_0+y_n+2\sum_{i=1}^{n-1}y_i}{2}h-\frac{M_0+M_n+2\sum_{i=1}^{n-1}M_i}{24}h^3
    \label{eq:cubic-int-equidistant}
\end{equation} 

Cubic一维积分程序见附录\ref{chap:code}。

\section{二维平面积分}
对二维笛卡尔坐标系$x-y$平面的积分,可先对$x$坐标积分,在对$y$坐标积分,这样可得到原函数在
二维平面上的积分。
\subsection{Cubic二维平面积分}
对于等间距的点,我们直接用式\eqref{eq:cubic-int-equidistant}来分别对$x$方向和$y$方向进行
积分。






%%%%%%%%%%%%%%%%%%%%%%%%%%%%%%%%%%%%%%%%%%%%%%%%%%%%%%%%%%%%%
% \chapter{数值微分}
% \section{五点数值微分公式}
% \subsection{二阶导微分公式}
% 设$f(x)$为定义在区间$[a,b]$上的函数,给定$f(x)$在等距点$a \leqslant x_0 < x_1 < x_2 < x_3
% < x_4 \leqslant b$,节点上函数值为$f(x_k)$,$(k=0, 1, 2, 3, 4)$,且满足$x_{k+1}-x_k=h$。
% 在$[a,b]$上做$f(x)$的4次Lagrange插值函数,并将$x=x_0+th, t \in [0, 4], x_k = x_0+kh$带入,
% 将方程两端对$t$求二次导,再分别把$t=0, 1, 2, 3, 4$带入,可得到$x_k$节点二阶导数的5点数值
% 微分公式,如下:
% \begin{equation}
% 	\begin{aligned}
% 		f^{\prime\prime}(x_0) =& \frac{1}{12h^2}[35f(x_0) - 104f(x_1) + 114f(x_2) - 56f(x_3) + 11f(x_4)] \\
% 		f^{\prime\prime}(x_1) =& \frac{1}{12h^2}[11f(x_0) -  20f(x_1) +   6f(x_2) +  4f(x_3) -   f(x_4)] \\
% 		f^{\prime\prime}(x_2) =& \frac{1}{12h^2}[ -f(x_0) +  16f(x_1) -  30f(x_2) + 16f(x_3) -   f(x_4)] \\
% 		f^{\prime\prime}(x_3) =& \frac{1}{12h^2}[ -f(x_0) +   4f(x_1) +   6f(x_2) - 20f(x_3) + 11f(x_4)] \\
% 		f^{\prime\prime}(x_4) =& \frac{1}{12h^2}[11f(x_0) -  56f(x_1) + 114f(x_2) -104f(x_3) + 35f(x_4)]
% 	\end{aligned}
% \end{equation} 




