\chapter{线性代数}

\section{行列式}

\subsection{余子式}
对$n$阶行列式$|\bm{A}|$,去掉其矩阵元$a_{ij}$所在行和列的所有元素后,留下的$n-1$阶行列式$|\bm{M}_{ij}|$为矩阵元$a_{ij}$对应的余子式。

\begin{example}
    $|\bm{A}|$是$3\times 3$的矩阵
    \begin{equation*}
    |\bm{A}| = \begin{vmatrix}
        a_{11} & a_{12} & a_{13} \\
        a_{21} & a_{22} & a_{23} \\
        a_{31} & a_{32} & a_{33}
    \end{vmatrix}
    \end{equation*}
    其矩阵元$a_{11}$对应的余子式为
    \begin{equation*}
    |\bm{M}_{ij}| = \begin{vmatrix}
        a_{22} & a_{23} \\
        a_{32} & a_{33}
    \end{vmatrix}
    \end{equation*}
\end{example}

%%%%%%%%%%%
\subsection{代数余子式}
矩阵元$a_{ij}$对应的代数余子式表示为
\begin{equation}
    |\bm{A}_{ij}| = (-)^{i+j} |\bm{M}_{ij}|
\end{equation}

\section{矩阵}

定义$\bm{I}$为单位矩阵
\begin{equation}
    \bm{I} = \begin{pmatrix}
        1 &   &          \\
          & 1 &          \\
          &   &  \ddots  \\
    \end{pmatrix}
\end{equation}

\paragraph*{矩阵的一些约定}
\begin{enumerate}
    \item 矩阵用粗体大写字母或圆括号包裹其中元素(对应的小写字母,下表表示对应的行和列)表示
        \begin{equation}
            \bm{A} = (a_{ij})
        \end{equation}
    \item 转置矩阵 $\bm{A}^{T}$ 或 $(a^{T}_{ij})$
    \item 逆矩阵 $\bm{A}^{-1}$ 或 $(a^{-1}_{ij})$。
    \item 伴随矩阵 $\bm{A}^{adj}$。
\end{enumerate}

%%%%%%%%%%%%
\subsection{伴随矩阵}
矩阵$\bm{A}$的伴随矩阵用其代数余子式表示为
\begin{equation}
    \bm{A}^{adj} = \begin{pmatrix}
        |\bm{A}_{11}| & |\bm{A}_{21}| & \cdots & |\bm{A}_{n1}| \\
        |\bm{A}_{12}| & |\bm{A}_{22}| & \cdots & |\bm{A}_{n2}| \\
        \vdots        & \vdots        & \ddots & \vdots        \\
        |\bm{A}_{1n}| & |\bm{A}_{2n}| & \cdots & |\bm{A}_{nn}|
    \end{pmatrix}
    \label{eq:adjoint-mat}
\end{equation}
其跟矩阵$\bm{A}$的关系为
\begin{equation}
    \bm{A}\bm{A}^{adj} = |\bm{A}|\bm{I}
    \label{eq:corr-mat-adjmat}
\end{equation}

%%%%%%%%%%%%
\subsection{逆矩阵}
矩阵$\bm{A}$的逆矩阵$\bm{A}^{-1}$满足
\begin{equation}
    \bm{A}\bm{A}^{-1} = \bm{A}^{-1}\bm{A} = \bm{I}
    \label{eq:inverse-mat}
\end{equation}
根据\cref{eq:corr-mat-adjmat,eq:inverse-mat},伴随矩阵与逆矩阵矩阵元$a^{-1}_{ij}$间的关系为
\begin{equation}\boxed{
        a^{-1}_{ij} = \frac{\bm{A}^{adj}_{ji}}{|\bm{A}|}
    \label{eq:corr-adjmat-invmat}
}\end{equation}

\section{本征值与本征矢}

\begin{definition}[本征值与本征矢]
  $\bm{A}$是$n$阶方阵,若存在数$\lambda$和非零矢量$\bm{x}$,使得$\bm{Ax}=\lambda \bm{x}$ ($\bm{x}\neq \bm{0}$),则$\lambda$为方阵$\bm{A}$的本征值,$\bm{x}$为对应的本征矢。
\end{definition}

\section{矩阵的迹}
矩阵的迹为其对角元之和,矩阵的迹又如下性质:
\begin{enumerate}
\item Linear property:
\begin{equation}
    {\rm Tr}(A+B) = {\rm Tr}(A) +{\rm Tr}(B)  \label{tracelinear}
\end{equation}

\item Multiplicity property:
\begin{equation}
    {\rm Tr}(cA) = c{\rm Tr}(A)  \label{tracemulti}
\end{equation}

\item Cyclic property:
\begin{equation}
  \begin{aligned}
    {\rm Tr}(AB) =& {\rm Tr}(BA)\\  \label{tracecyclic}
    {\rm Tr}[A,B]C =& {\rm Tr}A[B,C]
  \end{aligned}
\end{equation}
\end{enumerate}


\section{The Jacobi identity}
\begin{theorem}{The Jacobi identity}{}
  For three matrices $A, B$ and $C$, we have the identical relation
  \begin{equation}
    \left[A, [B,C]\right] + \left[C, [A,B]\right] + \left[B, [C,A]\right] = 0 \label{Jacobidentity}
  \end{equation}
\end{theorem}

%%%%%%%%%%%%%%%%%%%%%%
\section{行列式的导数}

\subsection{行列式对矩阵元的导数}
方阵$\bm{A}$对应的行列式为$|\bm{A}|$,它可以写成代数余子式$\bm{C_{ij}}$的求和
\begin{equation}
    |\bm{A}| = \sum_{j} a_{ij} |\bm{A}_{ij}|
    \label{eq:rep-cofactor}
\end{equation}
这样,该行列式对其中某一元素的导数为
\begin{equation}\boxed{
    \frac{\partial |\bm{A}|}{\partial a_{ij}} = \sum_{j} \frac{\partial \left(a_{ij} |\bm{A}_{ij}|\right)}{\partial a_{ij}} = |\bm{A}_{ij}|
    \label{eq:Det-part-elem}
}\end{equation}
上述表明行列式对某一元素的导数为其元素对应的代数余子式。

\subsection{行列式对其他变量的导数}
行列式中所有元素都是变量$t$的函数:$a_{ij}(t)$。由\cref{eq:Det-part-elem},根据变量$t$的求导用连续偏导方程展开
\begin{equation*}
    \frac{d |\bm{A}|}{d t} = \sum_{ij} \frac{\partial |\bm{A}|}{\partial a_{ij}} \frac{\partial a_{ij}}{\partial t} = \sum_{ij} |\bm{A}_{ij}| \frac{\partial a_{ij}}{\partial t} = |\bm{A}| \sum_{ij} \frac{|\bm{A}_{ij}|}{|\bm{A}|} \frac{d a_{ij}}{d t}
\end{equation*}
$\cfrac{d a_{ij}}{d t}$是$\cfrac{d \bm{A}}{d t}$的矩阵元,根据\cref{eq:corr-adjmat-invmat},上式最后一个等号改写为
\begin{equation*}
    |\bm{A}| \sum_{ji} (a)^{-1}_{ji} \left(\frac{d \bm{A}}{d t}\right)_{ij}
    =
    |\bm{A}| \sum_{j} \left( \bm{A}^{-1} \frac{d \bm{A}}{d t} \right)_{jj}
    = 
    |\bm{A}| {\rm Tr} \left( \bm{A}^{-1} \frac{d \bm{A}}{d t} \right)
\end{equation*}
所以最后有
\begin{equation}\boxed{
    \frac{d |\bm{A}|}{d t} = 
    |\bm{A}| {\rm Tr} \left( \bm{A}^{-1} \frac{d \bm{A}}{d t} \right)
    \label{eq:mat-der-var}
}\end{equation}

由\cref{eq:mat-der-var}可知,包含对数的求导$\cfrac{d \ln{|\bm{A}|}}{d t}$的求导过程为
\begin{equation}\boxed{
    \frac{\ln{|\bm{A}|}}{d t} = \frac{1}{|\bm{A}|} \frac{d |\bm{A}|}{d t} = {\rm Tr} \left( \bm{A}^{-1} \frac{d \bm{A}}{d t} \right)
    \label{eq:det-log-dev}
}\end{equation}
