\chapter{练习题}

\section{核物理练习题}

\begin{exercise}[Ex. C.1 in \citet{ningpz}]
    由式C1.21,比结合能写为
    \begin{equation}
        \frac{E_{B}}{A} =   \varepsilon_1 - \varepsilon_2 A^{-\frac{1}{3}}
                          - \varepsilon_3 \frac{Z^2}{A^{\frac{4}{3}}} 
                          - \varepsilon_4 \frac{(A - 2Z)^2}{A^2}
                          - \varepsilon_5 \frac{\delta}{A^{\frac{7}{4}}}
    \end{equation}
    将对应数值代入得$^{40}_{20}{\rm Ca}$得比结合能为
    \begin{equation*}
        \frac{E_{B}(20, 40)}{40} = 8.7177\, {\rm MeV}
    \end{equation*}
    同样可以得到
    \begin{equation*}
        \frac{E_{B}(64, 158)}{158} = 8.2381\, {\rm MeV}
        \quad
        \frac{E_{B}(100, 256)}{158} = 7.4656\, {\rm MeV}
    \end{equation*}
\end{exercise}

\vspace{3mm}
%%%%%%%%%%%%%%%%%
\begin{exercise}[Ex. C.2 in \citet{ningpz}]
    由式C1.21,原子核$^{A}{\rm Z}$得单中子分离能为
    \begin{equation*}
        \begin{aligned}
            S_n =& E_B(Z, A) - E_B(Z, A - 1)    \\
                =& \left[ \varepsilon_1 A - \varepsilon_2 A^{\frac{2}{3}}
                  - \varepsilon_3 \frac{Z^2}{A^{\frac{1}{3}}} 
                  - \varepsilon_4 \frac{(A - 2Z)^2}{A}
                  - \varepsilon_5 \frac{\delta}{A^{\frac{3}{4}}}\right] \\
                  &-
                  \left[ \varepsilon_1 (A - 1) - \varepsilon_2 (A - 1)^{\frac{2}{3}}
                  - \varepsilon_3 \frac{Z^2}{(A - 1)^{\frac{1}{3}}} 
                  - \varepsilon_4 \frac{(A - 1 - 2Z)^2}{A - 1}
                  - \varepsilon_5 \frac{\delta}{{(A - 1)}^{\frac{3}{4}}}\right] \\
                =& \varepsilon_1 - \varepsilon_2\left[A^{\frac{2}{3}} - (A - 1)^{\frac{2}{3}}\right]
                  -\varepsilon_3\left[\frac{Z^2}{A^{\frac{1}{3}}} - \frac{Z^2}{(A - 1)^{\frac{1}{3}}}\right] \\
                 & -\varepsilon_4\left[\frac{(A - 2Z)^2}{A} - \frac{(A - 1 - 2Z)^2}{A - 1} \right]
                  -\varepsilon_5\left[\frac{\delta}{A^{\frac{3}{4}}} - \frac{\delta}{{(A - 1)}^{\frac{3}{4}}}\right]
        \end{aligned}
    \end{equation*}
    对最后一个等号各项进行分别计算:\\
    1. 第二项,当$A$很大时,也就是对上式对$A\rightarrow \infty$求极限,有
    \begin{equation}
        \begin{aligned}
            \lim_{A \to \infty} \left(A^{\frac{2}{3}} - (A - 1)^{\frac{2}{3}} \right)
            &\xlongequal[]{A - 1 \to 1/x} \lim_{x \to 0} \left(\left(1 + \frac{1}{x}\right)^{\frac{2}{3}}
            - \left(\frac{1}{x}\right)^{\frac{2}{3}} \right) \\
            &= \lim_{x \to 0} \frac{(1 + x)^{2/3} - 1}{x^{2/3}} \\
            &= \lim_{x \to 0} \frac{(2/3) x}{x^{2/3}} \\
            &= \lim_{x \to 0} \frac{2}{3} x^{1/3}     \\
            &= \lim_{A \to \infty} \frac{2}{3} (A - 1)^{-1/3}
        \end{aligned}
    \end{equation}
    这里用到了等价无穷小$\lim_{x \to 0}(1 + x)^{n} \backsim n x$,第三个等号由于在$x = 0$的左领域不存在$x^{2/3}$的导数,
    故不能使用洛必达法则,我们再证
    \begin{equation*}
        \lim_{A \to \infty} \frac{(A - 1)^{-1/3}}{A^{-1/3}} = \lim_{A \to \infty}\left(1 - \frac{1}{A}\right)^{-1/3}
        = 1
    \end{equation*}
    可见当$A$很大时,$A^{-1/3}$与$(A - 1)^{-1/3}$是等价的,故此项可写为
    $$-\frac{2}{3}\varepsilon_2 A^{-1/3}$$
    2. 第四项,有
    \begin{equation*}
        \begin{aligned}
            \frac{(A - 2Z)^2}{A} - \frac{(A - 1 - 2Z)^2}{A - 1}
            =& \frac{(A - 2Z)^2(A - 1) - [(A - 2Z)^2 + 1 - 2(A - 2Z)]A}{A(A - 1)} \\
            =& \frac{A^2 - A - 4Z^2}{A(A - 1)} \\
            =& \frac{A}{A - 1} - \frac{1}{A - 1} - \frac{4Z^2}{A(A - 1)}
        \end{aligned}
    \end{equation*}
    当$A$很大时,有
    $$\lim_{A \to \infty} \frac{A}{A - 1} = 1$$
    $$\lim_{A \to \infty} \frac{1}{A - 1} = 0$$
    最后一项是考虑到$Z$和$A$始终保持在某一定比例内才能保证该核素能稳定存在,不考虑使用极限的方法进行化简。故此项最后化为
    $$-\varepsilon_4(1 - \frac{4Z^2}{A(A - 1)})$$
    3. 第三项和第五项,直接有
    \begin{equation}
        \begin{aligned}
            \lim_{A \to \infty} \left(\frac{Z^2}{A^{1/3}} - \frac{Z^2}{(A - 1)^{1/3}}\right) = 0 \\
            \lim_{A \to \infty} \left(\frac{\delta}{A^{3/4}} - \frac{\delta}{(A - 1)^{3/4}}\right) = 0
        \end{aligned}
    \end{equation}
    故对大质量的核素,其单中子分离能可近似表示为
    $$S_n = \varepsilon_1 - \frac{2}{3}\varepsilon_2 A^{-1/3} - \varepsilon_4(1 - \frac{4Z^2}{A(A - 1)})$$
\end{exercise}

\vspace{3mm}
%%%%%%%%%%%%%%%%
\begin{exercise}[Ex. C.14 in \citet{ningpz}]
    $\ket{jj}$对应的单粒子波函数为
    \begin{equation*}
        \phi_{nljm} = R_{nl}(r) \sum_{m, m_{S}} \Braket{l\,\nicefrac{1}{2}\,m\,m_{S} | j\,j} Y_{lm}(r, \theta, \phi) \chi_{\frac{1}{2}, m_{S}}
    \end{equation*}
    则对应的四极矩写为
    \begin{equation*}
        \begin{aligned}
            Q =& \sqrt{\frac{16\pi}{5}} e \Braket{jj| r^2 Y_{20}(\theta, \phi) | jj} \\
            =&\sqrt{\frac{16\pi}{5}} e 
            \int d\,\tau \left[\left(R_{nl}(r) \sum_{m^{\prime}, m_{S}^{\prime}} \Braket{l\,\nicefrac{1}{2}\,m^{\prime}\,m_{S}^{\prime} | j\,j} Y_{lm^{\prime}}(r, \theta, \phi) \chi_{\frac{1}{2}, m_{S}^{\prime}}\right)^{\ast}
            r^2 Y_{20}(\theta, \phi) \right.\\
            & \cdot \left. \left(R_{nl}(r) \sum_{m, m_{S}} \Braket{l\,\nicefrac{1}{2}\,m\,m_{S} | j\,j} Y_{lm}(r, \theta, \phi) \chi_{\frac{1}{2}, m_{S}}\right) \right] \\
            =& \sqrt{\frac{16\pi}{5}} e
              \int r^2 R_{nl}^{\ast}(r) R_{nl}(r)\, r^2dr \\
             &\cdot
               \sum_{mm^{\prime}m_{S}m_{S}^{\prime}} \int_{0}^{\pi}\sin{\theta}\, d\theta \int_{0}^{2\pi} d\phi \Braket{l\nicefrac{1}{2}m^{\prime}m_{S}^{\prime} | jj}\Braket{l\nicefrac{1}{2}mm_{S} | jj}
               Y_{lm^{\prime}}^{\ast}(r, \theta, \phi) Y_{20}(\theta, \phi) Y_{lm}(r, \theta, \phi) \chi_{\frac{1}{2}m_{S}^{\prime}} \chi_{\frac{1}{2}m_{S}}\\
        \end{aligned}
    \end{equation*}
    考虑到
    \begin{equation*}
        \chi_{\frac{1}{2}m_{S}^{\prime}} \chi_{\frac{1}{2}m_{S}} = \delta_{m_{S}m_{S}^{\prime}}
    \end{equation*}
    ,CG系数要求满足$m_S + m = j_z$才不为0,这里确定$j_z = j$, $m_S = m_S^{\prime}$也确定,那么有$m = m^{\prime}$。故去掉上式的两个哑指标$m^{\prime}$、$m^{\prime}_{S}$,将两个CG系数提到积分号外,同时$Y^{\ast}_{lm} = (-1)^{m}Y_{l-m}$,故上式写成
    \begin{equation*}
        \begin{aligned}
        Q =& \sqrt{\frac{16\pi}{5}} e
           \int r^2 R_{nl}^{\ast}(r) R_{nl}(r)\, r^2dr \\
           \cdot
           &\sum_{m m_{S}}  \Braket{l\frac{1}{2}mm_{S} | jj}^2  (-1)^{m}
           \int_{0}^{\pi}\sin{\theta}\, d\theta \int_{0}^{2\pi} d\phi
           Y_{l-m}(r, \theta, \phi) Y_{20}(\theta, \phi) Y_{lm}(r, \theta, \phi) \\
          =&  \sqrt{\frac{16\pi}{5}} e
           \int r^2 R_{nl}^{\ast}(r) R_{nl}(r)\, r^2dr
           \cdot
           \sum_{m m_{S}}  \Braket{l\frac{1}{2}mm_{S} | jj}^2 (-1)^{m}
           \sqrt{\frac{5(2l + 1)^2}{4\pi}}
           \begin{pmatrix}
               l & 2 & l \\
               0 & 0   & 0
           \end{pmatrix}
           \begin{pmatrix}
               l & 2 & l \\
               0 & 0 & 0
           \end{pmatrix} \\
           =&  \sqrt{\frac{16\pi}{5}} e
           \int r^2 R_{nl}^{\ast}(r) R_{nl}(r)\, r^2dr
           \cdot \sqrt{\frac{5}{4\pi}}
           \sum_{m m_{S}}
           \Braket{l\frac{1}{2}mm_{S} | jj}^2
           \Braket{l200 | l0}
           \Braket{l2m0 | lm} \\
           =&  2 e
           \int r^2 R_{nl}^{\ast}(r) R_{nl}(r)\, r^2dr
           \cdot
           \sum_{m m_{S}}
           \Braket{l\frac{1}{2}mm_{S} | jj}^2
           \Braket{l200 | l0}
           \Braket{l2m0 | lm} \\
        \end{aligned}
    \end{equation*}
    上式第三个等号从3j符号转到CG系数用到书籍\textit{Quantum Theory of Angular Momentum}中Section 8.1.2的Eq.\,(11);同时用到书籍中Tab.8.1 $\sim$  8.10的CG系数公式,先计算求和符号的后两项
    \begin{equation*}
        \Braket{l\, 2\, 0\, 0 | l\, 0}\Braket{l\, 2\, m\, 0 | l\, m} 
        =
        \frac{-l(l+1)}{\left[(2l - 1)l(l+1)(2l+3)\right]^{\nicefrac{1}{2}}}
        \cdot
        \frac{3l^2 - l(l+1)}{\left[(2l - 1)l(l+1)(2l+3)\right]^{\nicefrac{1}{2}}}
    \end{equation*}
    以下分两种情况进行讨论
    \begin{enumerate}
        \item 
    \end{enumerate}
\end{exercise}

%%%%%%%%%%%%%%%%
\begin{exercise}[Ex 2.25 in \citet{suhonen-NtoN}]
    According to the Winger-Eckart theorem(Eq. 2.27 in the book), 
    the reduced probability can be derived from
    \begin{equation}
    \begin{aligned}
        B(\sigma\lambda; J \rightarrow J^{\prime})
        &\equiv
        \sum_{\mu M^{\prime}} \left|
            \Braket{\alpha^{\prime} J^{\prime} M^{\prime} |
            \mathcal{M}_{\sigma\lambda\mu} | \alpha J M}
        \right|^2 \\
        &= \sum_{\mu M^{\prime}} \left|
        (-1)^{j^{\prime} - M^{\prime}}\begin{pmatrix}
            J^{\prime}  &  \lambda  &  J \\
            -M^{\prime} &  \mu      &  M 
        \end{pmatrix}
        \right|^2 \left|
          \left(\alpha^{\prime} J^{\prime} \| \mathcal{M}_{\sigma\lambda} \|  \alpha J\right)
        \right|^2 \\
        &= \left|
            \left(\alpha^{\prime} J^{\prime} \| \mathcal{M}_{\sigma\lambda} \|  \alpha J\right)
          \right|^2
          \sum_{\mu M^{\prime}} \left|\begin{pmatrix}
            J^{\prime}  &  \lambda  &  J \\
            -M^{\prime} &  \mu      &  M 
        \end{pmatrix}
        \right|^2  \\ 
        &= \left|
            \left(\alpha^{\prime} J^{\prime} \| \mathcal{M}_{\sigma\lambda} \|  \alpha J\right)
          \right|^2
          \sum_{\mu M^{\prime}} \left|
            (-1)^{J^{\prime} - J - M} \hat{J}^{-1}
            \braket{J^{\prime} -M^{\prime} \lambda \mu | J -M}
        \right|^2  \\
        &\xlongequal[]{\text{see Eq. 8-(8) in Ref. \cite{varshalovich-amt}}}
        \hat{J}^{-1} \left|
            \left(\alpha^{\prime} J^{\prime} \| \mathcal{M}_{\sigma\lambda} \|  \alpha J\right)
          \right|^2\\
          &=  \frac{1}{2J+1} \left|
            \left(\alpha^{\prime} J^{\prime} \| \mathcal{M}_{\sigma\lambda} \|  \alpha J\right)
          \right|^2
    \end{aligned}
    \end{equation}
    the sum symbol in fourth equal sign, the CG coefficient satisfies
    \begin{equation*}
        \left\{
            \begin{array}{c}
                \Delta(J^{\prime} \lambda J) \\
                -M^{\prime} + \mu = -M
            \end{array}
        \right.
    \end{equation*}
    so the index $\mu$ can be replaced by $M$ and the completeness
    property of CG coefficient Eq. 1.28 can be use.
    \label{exer-suhonen:ex2-25}
\end{exercise}

%%%%%%%%%%%%%%
\begin{exercise}[Ex 2.25 in \citet{suhonen-NtoN}]
    由于讨论的是单粒子的矩阵元,因此\cref{eq:mul-elec}中的求和号可去掉,变成只对一个核子的计算:
    $$Q_{\lambda\mu} = \zeta^{(E\lambda)} e r^{\lambda} Y_{\lambda\mu}(\Omega)$$
    根据前面对单粒子态的定义\cref{eq:sgl-part-stat}:$a \equiv n_{a} l_{a} j_{a}$,再次,由\citet[see][Eq. 2.57]{suhonen-NtoN}
    \begin{equation*}\begin{aligned}
        \Braket{a || \bm{Q}_{\lambda} || b } =& \Braket{n_a l_a j_a || \bm{Q}_{\lambda} || n_b l_b j_b} \\
        =& \Braket{n_a l_a j_a || \zeta^{(E\lambda)} e r^{\lambda} \bm{Y}_{\lambda} || n_b l_b j_b} \\
        =& \zeta^{(E\lambda)} e \Braket{n_a l_a j_a || r^{\lambda} \bm{Y}_{\lambda} || n_b l_b j_b} \\
        =& \zeta^{(E\lambda)} e \underbrace{\Braket{n_a l_a || r^{\lambda} || n_b l_b}}_{\rm radial} \underbrace{\Braket{ l_a j_a || \bm{Y}_{\lambda} || l_b j_b}}_{\rm angular} \\
        =& \zeta^{(E\lambda)} e \cdot
           \int_0^{\infty} g_{n_a l_a}(r) r^{\lambda} g_{n_b l_b}(r) r^2\, dr \cdot
           \frac{1}{\sqrt{4\pi}} (-1)^{j_b - \nicefrac{1}{2} + \lambda} \frac{1 + (-1)^{l_a + l_b + \lambda}}{2} \hat{j_a}\hat{j_b}\hat{\lambda}\begin{pmatrix}
            j_a         &           j_b  & \lambda \\
            \frac{1}{2} &   -\frac{1}{2} &   0
           \end{pmatrix} \\
        =& \zeta^{(E\lambda)} \frac{e}{\sqrt{4\pi}}
            (-1)^{j_b - \nicefrac{1}{2} + \lambda} \frac{1 + (-1)^{l_a + l_b + \lambda}}{2} \hat{j_a}\hat{j_b}\hat{\lambda}\begin{pmatrix}
            j_a         &           j_b  & \lambda \\
            \frac{1}{2} &   -\frac{1}{2} &   0
           \end{pmatrix}
           \mathcal{R}_{ab}^{(\lambda)}  
    \end{aligned}\end{equation*}
    其中,第四个等号将量子数标记的态利用分离变量的方法化成了径向和角向两部分,并在最后一个等号定义了
    $$\mathcal{R}_{ab}^{(\lambda)} \equiv \int_0^{\infty} g_{n_a l_a}(r) r^{\lambda} g_{n_b l_b}(r) r^2\, dr$$
\end{exercise}


\bibliographystyle{modified-apsrev4-2}
\bibliography{reference}