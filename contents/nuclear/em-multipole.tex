\chapter{电磁多极矩和跃迁}

\section{电磁可观测量的一般性质}
原子核电磁场衰变是原子核和外部电磁场相互作用的结果。外部电磁场由电场$\bm{E}$和磁场
$\bm{B}$组成,电磁场能量正比于$\bm{E}^2 + c^2 \bm{B}^2$。用势$(\phi, \bm{A})$作为原子核
和场相互作用的媒介。其中$\phi$是标量势,与原子核密度相关,表示电部分的相互作用;
$\bm{A}$是标量势,与原子核流密度相关,表示磁部分的相互作用。

\CJKunderline{原子核流密度(current density)组成:运动的质子构成轨道部分,质子和
中子的自旋构成自旋部分}。

电磁辐射场可展开成多极形式(球谐函数展开),并用声子进行量子化。原子核加上场才是
完整的系统,这两部分的相互作用很弱,一般用微扰理论进行处理。考虑一个处于激发态
的原子核通过电磁衰减至基态,那么无微扰的系统初态是激发的原子核态加上处于基态的
电磁场;系统末态是原子核基态加上电磁场的1-声子态。从初态到末态的跃迁由辐射场展
开的多极项中的一个作为媒介来完成。 

\subsection{跃迁概率和半衰期}
\begin{definition}[跃迁概率]
    从原子核初态$i$通过$\gamma$-衰变跃迁到末态$f$的每单位时间的跃迁概率,记作$T_{fi}$。
\end{definition}
\begin{itemize}
    \item 跃迁概率对应的\CJKunderdot{寿命}为$1/T_{fi}$。
    \item 对应的\CJKunderdot{半衰期}为
    \begin{equation}
        \boxed{
        t_{1/2} = \frac{\ln{2}}{T_{fi}}\ .
        }
    \end{equation}
\end{itemize}

用指标$\sigma=\,$E or M指代电或磁的场类型,则对应的$\sigma\lambda\mu$跃迁概率计算公式为
\begin{equation}
    T_{fi}^{(\sigma\lambda\mu)} = \frac{2}{\epsilon_0\hbar}\frac{\lambda+1}{\lambda\left[(2\lambda+1)!!\right]^2}\left(\frac{E_{\gamma}}{\hbar c}\right)^{2\lambda+1} \left|\Braket{\xi_{f}J_{f}m_{f} | \mathcal{M}_{\sigma\lambda\mu} | \xi_{i} J_{i} m_{i}}\right|^2
\end{equation}
其中,$E_{\gamma}$为跃迁能量,$\mathcal{M}_{\sigma\lambda\mu}$为场$\sigma\lambda\mu$多极辐射对应的原子核算符。

\CJKunderdot{约化跃迁概率}(推导详见Ex. \ref{exer-suhonen:ex2-25})为
\begin{equation}
    B(\sigma\lambda; \xi_i J_i \rightarrow \xi_f J_f) \equiv \frac{1}{2J_i + 1} \left|\Braket{\xi_f J_f || \mathcal{M}_{\sigma\lambda} || \xi_i J_i}\right|^2
    \label{eq:reduced-TransProb}
\end{equation}
电和磁张量算符的分量为
\begin{align}
    Q_{\lambda\mu} &= \zeta^{(E\lambda)} \sum_{j = 1}^{A} e(j) r_j^{\lambda} Y_{\lambda\mu}(\Omega_{j}) \label{eq:mul-elec} \\
    M_{\lambda\mu} &= \frac{\mu_{N}}{\hbar c} \zeta^{(M\lambda)} \sum_{j = 1}^{A} \left[\frac{2}{\lambda+1}g_{l}^{(j)} \bm{l}(j) + g_{s}^{(j)}\bm{s}(j)\right] \cdot \nabla_j [r_j^\lambda Y_{\lambda\mu}(\Omega_j)]    \label{eq:mul-magn}
\end{align}
其中,$j$表示第$j$个核子,$e(j)$表示第$j$个核子的电荷,$\bm{l}(j)$和$\bm{s}(j)$分别该核子的轨道和自旋角动量。以下,$e$表示质子的单位电荷,中子电荷为0。

%%%%%%%%%%%%%%%%%%%
\subsection{选择定则}
选择定则包含\CJKunderdot{宇称选择定则}和\CJKunderdot{角动量守恒定则}。
\begin{question}[一个常数无法连接两个不同的原子核态。]\label{ques:const-not-connect-stat}
    将一个常数取代$\mathcal{M}_{\sigma\lambda}$放进约化跃迁概率\cref{eq:reduced-TransProb}中,由于常数可以提出去,最后只剩下$\Braket{\xi_f J_f |\xi_i J_i} = 0$(不同态的正交性),因此一个常数无法给出两个态之间的关系。
\end{question}
\begin{enumerate}
    \item \uline{没有\mbox{$E0$}和\mbox{$M1$}跃迁}:由\cref{eq:mul-elec},$\lambda=0$时,此张量形式为一个常数,由\cref{ques:const-not-connect-stat}的论断,$E0$跃迁不存在;由\cref{eq:mul-magn},$\lambda=0$时,由于$\nabla$的存在,其对$r^0$求导为0,故$M_0$跃迁不存在。
\end{enumerate}

%%%%%%%%%%%%%%%%%%%%%%%%%%%%%%%%%%%%%%%%%%%%%
\section{One-Particle和One-Hole原子核的电磁跃迁}

%%%%%%%%%%%%%%%%%%%%%%
\subsection{约化跃迁概率}
\paragraph*{One-Particle单体跃迁概率}
One-Particle的初态和末态表示为
\begin{align}
    \ket{\Psi_i} =& \ket{n_i l_i j_i m_i} = c_{i}^{\dagger} \ket{\rm{CORE}} \\
    \ket{\Psi_f} =& \ket{n_f l_f j_f m_f} = c_{f}^{\dagger} \ket{\rm{CORE}}
\end{align}
可得到单体跃迁密度为
\begin{equation}
    \begin{aligned}
    \Braket{\Psi_f | [c_a^{\dagger} c_b]_{\lambda\mu} | \Psi_i}
    =&
    \sum_{m_\alpha m_\beta} \left(j_a m_\alpha j_b m_\beta | \lambda \mu\right)
    \Braket{\rm{CORE} | c_f c_\alpha^\dagger \tilde{c}_\beta c_i^\dagger | \rm{CORE}} \\
    =& \delta_{af}\delta_{bi} (-1)^{j_i - m_i} \left(j_f m_f j_i -m_i | \lambda \mu\right)        
    \end{aligned}
\end{equation}
\begin{proof}
    根据\Cref{eq:sing-reduced-part-elem},对$[c_a^{\dagger} c_b]_{\lambda\mu}$进行展开,有
    \begin{equation}
        [c_a^{\dagger} c_b]_{\lambda\mu} = 
    \end{equation}
\end{proof}
