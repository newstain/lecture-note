\chapter{物质中的电场}

\section{极化}

\paragraph*{电介质}
大多数物质(在很好的近似下)可划分为两大类:导体和绝缘体(或电介质)。

导体是可“无限制”的提供在其内部可自由移动的电荷物体,如各种金属,其实质是许多电子并不
和其内部任何特定的原子核关联,可在导体中漫游。电介质中的电荷依附于特定的原子和分子,
他们被束缚着,仅能在原子或分子中移动较小的位移,因此无法像导体一样,但其累积效应可以
解释电介质材料的行为特性。

\paragraph*{诱导偶极子}
中性原子置于外部电场$E$时,其带负电的电子向电场方向移动,带正电的原子核向电场负方向
移动导致导致正负电荷质心不再重叠,使原子发生极化,产生一个微小的偶极矩$\bm{p}$,其与
电场方向$\bm{E}$相同,且近似与改电场成正比:
\begin{equation}
    \bm{p} = \alpha \bm{E}
    \label{eq:charge-polorize}
\end{equation}
式中的比例常数$\alpha$称为原子极化率。
